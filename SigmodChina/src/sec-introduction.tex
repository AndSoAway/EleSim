%!TEX root=../sig-alternate-sample.tex

\section{Introduction}
China has gone through a rapid economic growth in the past three decades, and economy boosting in east china area has been always a highlight. Ever since, electricity has been a powerful catalyst for social development and economic growth. Empirically, economic fluctuation would cause changes in electricity consumption and electricity consumption can also reflect the economic change. There has been much interests in the causal relationship between economic growth and electricity consumption using the regression or statistical method. However, though the economic indicators and electricity consumption can be viewed as time series and lots of trajectory similarity methods exist, a big gap still lies midst them. Our work in this paper would focus on the exploit of internal link between economic indicators and electricity consumption using trajectory similarity.

Nowadays, the GDP growth has slower to blew 7\%, which officially is called "new normal". Economic policies would adjust to this new situation, thus asking energy strategies to change. Consequently, the study of the casual relationship between economic indicators and the electricity consumption would shed light on future energy policies, such as the energy conservation, the planning of capacity expansion and the reliable supply of electricity.
  
There are numerous economic indicators and previous work mainly focused on the simple GDP. The study of the relationship between energy consumption and economic growth started with the seminal work of \cite{kraft:relationship}, in which causality between the energy consumption and GNP growth was found using Granger test. Subsequently, actual circumstances in the developed countries was studied like the United Kingdom, Germany, Italy\cite{yu:causal, erol1987causal}. Due to the boosting prosperity, Asian economy also attracted the eyes of researchers like South Korea and Singapore. The cointegration and error-correction models were also applied to research from the viewpoint of the time-series\cite{glasure1998cointegration}.

However, previous work just treat the data as 

Compare to the previous work, our paper has made the following contributions.
\begin{itemize}
	\item We proposed an similarity metrics that 
	\item We focused on specific provinces in the east China.
	\item Quantitive indicators are tested and the most closely connected are sought.
\end{itemize}

The rest part of this paper is organized as follows. We formalize our question in section 2 and introduce our algorithms in section 3. In section 4, we will elaborate our experiments and highlight our contributions. At last, the section 5 will give the conclusion and the future work.
