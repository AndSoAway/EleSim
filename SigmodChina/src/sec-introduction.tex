%!TEX root=../sig-alternate-sample.tex

\section{Introduction}
China has gone through a rapid economic growth in the past three decades, and economy boosting in east china area has always been a highlight. Ever since, electricity has been a powerful catalyst for social development and economic growth. Empirically, economic fluctuation would cause changes in electricity consumption and electricity consumption can also reflect the economic change. Nowadays, the GDP growth has slower to blew 7\%, which officially is called "new normal". Economic policies would adjust to this new situation, thus asking energy strategies to change. New situation has raised many command for the research to the relationship between the economic indicators and the electricity consumption. 

Historically, there has been lots of work in the causal relationship between economic growth and electricity consumption, most of which used the regression or statistical method traditionally. Many previous work mainly focused on the simple GDP. The study of the relationship between energy consumption and economic growth started with the seminal work of \cite{kraft:relationship}, in which causality between the energy consumption and GNP growth was found using Granger test. Subsequently, actual circumstances in the developed countries was studied like the United Kingdom, Germany, Italy\cite{yu:causal, erol1987causal}. Due to the boosting prosperity, Asian economy also attracted the eyes of researchers like South Korea and Singapore. The cointegration and error-correction models were also applied to research from the viewpoint of the time-series\cite{glasure1998cointegration}.

But these works cannot focus the similarity of tendency treat the economic development process as a whole. Meanwhile, not only the GDP or GNP but also many other economic indicators in many fields such as industry, investment, etc, have strong ties with the electric power metrics. Also, traditional work researched the whole country, but we all know that regional developmental difference could be so astonishingly large that the result could not reflect the true closeness between electricity and economy.  

From another viewpoint, the economic indicators and electricity consumption can be viewed as time series data, which are similar to the trajectory. A traditional trajectory is a sequence of sample points consisting of geo-location and timestamp. Similarly, both the economic indicators and electricity consumption data consists of the value and timestamp. Meanwhile, we can compare trajectory similarity using many methods focusing on different advantages. Our work in this paper would focus on the exploit of internal link between economic indicators and electricity consumption using trajectory similarity, which is much novel comparing to the previous work on the relationship between economic indicators and electricity consumption. 

To address this problem, we propose a trajectory-similarity framework, which allows us to select the top-k closely relevant indicators using the trajectory similarity and verify the lagging time between indicators and electricity consumption. Firstly, we normalize the indicator data and electricity data to make them comparable, and then . Compare to the previous work, our paper make the following contributions.
\begin{itemize}
	\item We proposed an similarity metrics that
	\item We focused on specific provinces in the east China.
	\item Quantitive indicators are tested and the most closely connected are sought.
\end{itemize}

The rest part of this paper is organized as follows. We formalize our question in section 2 and introduce our algorithms in section 3. In section 4, we will elaborate our experiments and highlight our contributions. At last, the section 5 will give the conclusion and the future work.

Consequently, the study of the casual relationship between economic indicators and the electricity consumption would shed light on future energy policies, such as the energy conservation, the planning of capacity expansion and the reliable supply of electricity.