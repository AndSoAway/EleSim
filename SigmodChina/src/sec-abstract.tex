%!TEX root=../sig-alternate-sample.tex


\begin{abstract}
As is widely known, the electricity industry is closely connected to the economic development. Lots of work has focused on the regression relationship between the economic metrics and electricity-related indexes. Studies on this relationship have great significance on economic predication and regulation. However, most of the previous work focused on the regression. Obviously, the economic and electricity data can be treated as time series data, which is called $trajectory$ in this paper. The similarity of two trajectory could reflect how much two metrics change in the same tendency. We proposed a new trajectory similarity, which based on the area enclosed by two trajectory. Different to the traditional polygon area, we keep one trajectory stay, and move another trajectory vertically seeking for the least area. Meanwhile, the some  economic activities would last for some time. So, there exists some time lagging between the economic activities and the electricity consumption. We could move the economic indicators data horizontally so that we could get the most similarity. Therefore, we could get the time offset between the economic consumption and economic indicators. In the experiment, We have taken another three similarity function to compare the results. 
\end{abstract}

\keywords{Electricity Consumption; Economic Indicator; Similarity}
