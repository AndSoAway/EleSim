%!TEX root=../sig-alternate-sample.tex

\begin{figure}[!t]
\linesnumbered \SetVline \setcounter{algocf}{0}

\begin{algorithm}[H]
\label{alg:areasim} \caption{AreaSim($T$, $Q$)}
\KwIn{
 	$T$: economic indicator trajectory \\
 	\hspace{3.6em}$Q$: electricity consumption trajectory \\
}
\KwOut{$\epsilon$: similarity of $T$ and $Q$}
\SetVline
\Begin{
$\epsilon$ = 0\;
\For{ $p^i_Q \in Q$} 
{
	dis = p^i_T - p^i_Q;\\
	Q^\prime = \For{$p^i_Q \in Q$} {p^i_Q + dis;\\}
	$\epsilon^{\prime} = AreaSim(T, Q^\prime)$;\\
	$\epsilon = \max(\epsilon, \epsilon^{\prime})$;
}
\Return{$\epsilon$;} \nllabel{alg:areasim:return} 
}
\end{algorithm}

\begin{algorithm}[H]\label{alg:lagging}
%\nonumber
%\linesnumbered \dontprintsemicolon
\KwIn{
	$T$: economic indicator trajectory \\
	\hspace{3.6em}$Q$: electricity consumption trajectory \\
}	
\KwOut{	$t$: lagging time between $T$ and $Q$}
\SetVline	
\Begin{
\epsilon = 0, $t$ = -1; \\
Choose part of $T$ as $T^{\prime}$\;
\For{i \in (-12:12)} {
	add i to date of each point in $Q$;\\
	choose part of $Q$ as $Q^{\prime}$, which has same date with $T^{\prime}$;\\
	\epsilon^{\prime} = AreaSim(T^{\prime}, Q^{\prime});\\
	\If{\epsilon^{\prime} > \epsilon } {
		\epsilon = \epsilon^{\prime};\\
		$t$ = i;\\
	}\\
}
\Return{$t$} \caption{Lagging($T$, $Q$)}
}
\end{algorithm}
\caption{AreaSim FrameWork.}\label{algo:framework}
\vspace{-1.5em}
\end{figure}